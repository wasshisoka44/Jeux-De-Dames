%pdflatex -halt-on-error -aux-directory=tmp -output-directory=tmp rapport.tex%

\documentclass{article}
\usepackage{amsmath}
\usepackage[utf8]{inputenc}
\usepackage[T1]{fontenc}
\usepackage{graphicx}
\usepackage{hyperref}
\usepackage[francais]{babel}
\usepackage{listings}


\title{Rapport sur le jeu de damme}
\author{Wassim Saidane, Aurélien Authier}
\date{}

\begin{document}
    \lstset{language=C}
    \pagenumbering{gobble}
    \maketitle
    \tableofcontents
    \newpage
    \pagenumbering{arabic}
    \section{Introduction}

    \section{Protocole choisis}
        Nous avons choisis d'utiliser le protocole TCP car garantit l’émission ainsi que la réception de paquets de données.
    \section{Implémentation}
    \subsection{Serveur}
    Nous avons repris le code du TP5 permettant de d'établir un tchat entres les clients.
    Le nombre de client est prédéfinit. 
    \begin{lstlisting}[frame=single]  
        #define MAX_CLIENTS 2
    \end{lstlisting}
    Le client est définit par une structure
    \begin{lstlisting}
        typedef struct client_s
            {
                struct sockaddr caddr;
                socklen_t clen;
                int cs;
                char buf[BUF_SIZE + 1];
            } client_t;
    \end{lstlisting}
    On définit une liste de client et la taille du message (déplacement du piont du joueurs).
    \begin{lstlisting}
        client_t clients[MAX_CLIENTS] = {0};
        ssize_t msg_len;
    \end{lstlisting}
    Le serveur affiche le déplacement du joueur comme ceci
    \begin{lstlisting}[frame=single]
        printf("Le joueur %d deplace le pion %s\n",
         j + 1, clients[j].buf);
    \end{lstlisting}

    
    \subsection{Client}
    On va d'abord définir 3 strings (pion,x,y) : 
    \begin{lstlisting}
        char pion[BUF_SIZE + 1];
        char x[BUF_SIZE + 1];
        char y[BUF_SIZE + 1];
        bzero(pion, BUF_SIZE + 1);
        bzero(x, BUF_SIZE + 1);
        bzero(y, BUF_SIZE + 1);
        ssize_t msg_len;
    \end{lstlisting}
    pion est le numéro du pion, x et y ses coordonnées.
    \newpage
    Ensuite dans une boucle on demande au joueur de choisir le pion à déplacer ainsi que ses nouvelles coordonnées.
    \begin{lstlisting}
        while (1)
        {
            bzero(pion, BUF_SIZE + 1);
            printf("Quel pion voulez vous bougez ? : ");
            fgets(pion, BUF_SIZE, stdin);
            if (!strlen(pion))
                break;
            printf("Vous deplacez le pion %s", pion);

            bzero(x, BUF_SIZE + 1);
            printf("Nouvelle coordonnee de x : ");
            fgets(x, BUF_SIZE, stdin);

            bzero(y, BUF_SIZE + 1);
            printf("Nouvelle coordonnee de y : ");
            fgets(y, BUF_SIZE, stdin);

            printf("Le pion %s a pour nouvelle coordonnee [%s,%s]", pion, x,y);
        }
    \end{lstlisting}
\end{document}