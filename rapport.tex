%pdflatex -halt-on-error -aux-directory=tmp -output-directory=tmp rapport.tex%

\documentclass{article}
\usepackage{amsmath}
\usepackage[utf8]{inputenc}
\usepackage[T1]{fontenc}
\usepackage{graphicx}
\usepackage{hyperref}
\usepackage[francais]{babel}
\usepackage{listings}


\title{Rapport sur le jeu de damme}
\author{Wassim Saidane, Aurélien Authier}
\date{}

\begin{document}
    \lstset{language=C}
    \pagenumbering{gobble}
    \maketitle
    \tableofcontents
    \newpage
    \pagenumbering{arabic}
    \section{Introduction}

    \section{Protocole choisis}
        Nous avons choisis d'utiliser le protocole TCP car garantit l’émission ainsi que la réception de paquets de données.
    \section{Implémentation}
    \subsection{Serveur}
    Nous avons repris le code du TP5 permettant de d'établir un tchat avec les clients.
    Le plateau et un objet de structure pion (voir 3.3) sont initialisé.
    \begin{lstlisting}
        int sockfd, len, connfd[2], plateau[TAILLE * TAILLE];
        Pion p;
    \end{lstlisting}
    On demande aux clients le pion qu'ils veulent déplacer 
    \begin{lstlisting}
        char buf[100] = "Quel pion voulez vous deplacer ? ";
    \end{lstlisting}
    Le plateau est ensuite initialisé, le message est envoyé au client et le serveur affiche le déplacement
    \begin{lstlisting}
        init_game(plateau);
        while (1)
        {
            for (int i = 0; i < 2; i++)
        {
            write(connfd[i], (const char *)&plateau, sizeof(plateau));
            printf("Le Joueur %d depalce un piont\n", i + 1);
            write(connfd[i], (const char *)&buf, sizeof(buf));
            recv(connfd[i], &p, sizeof(p), 0);
            printf("Le joueur %d a deplacer le pion 
            :\n x : %d\ny : %d\n", i + 1, p.x, p.y);
        }
    }
    \end{lstlisting}

    
    \subsection{Client}
    
    \subsection{Le jeu}
\end{document}